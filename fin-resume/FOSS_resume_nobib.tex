
\hyphenation{pdfTeX Mathematica}

\documentclass{article}

% capitalize section titles; center
\makeatletter
\renewcommand\section{\@startsection {section}{1}{\z@}%
                                   {-1.5ex \@plus -1ex \@minus -.2ex}%beforeskip
                                   {1.0ex \@plus.2ex}%afterskip
                                   {\Large\scshape\centerline}*}
\makeatother

\setlength{\parindent}{0cm}

\newcommand\listInd{0.4cm}

\usepackage{url}
\usepackage[inner=1.0in,outer=1.0in,bottom=2.5cm,top=2.5cm]{geometry}

%\usepackage[backend=biber,style=authoryear,maxbibnames=99,maxcitenames=99]{biblatex}
%\addbibresource{../biblio/resumeBiblio.bib}


%\textwidth=5.2in % increase textwidth to get smaller right margin
%\usepackage{helvetica} % uses helvetica postscript font (download helvetica.sty)
%\usepackage{newcent}   % uses new century schoolbook postscript font 

\begin{document} 

\thispagestyle{empty}
\pagestyle{empty}  
 
\begin{centering}
\Large
Alexander H. Foss
\vspace{0.05in}
\normalsize

\hspace*{\fill}
%first item
alexanderhfoss@gmail.com
\hfill
\textbullet
\hfill
% second item 
(931) 636-8109 
\hfill
\textbullet
\hfill
% third item
linkedin.com/in/alex-foss-30939383
%\hfill
%\textbullet
%\hfill
\hspace*{\fill}

\vspace{-0.08in}

\end{centering}


\noindent\makebox[\linewidth]{\rule{\textwidth}{0.4pt}}

\vspace{-0.05in}

 
\section{Summary}
I am an inquisitive, effective, and creative statistician and data scientist with extensive skills in three broad areas.
\textbf{Statistics and machine learning}, including experimental design, model building, linear/nonlinear regression, prediction, hypothesis testing, classification, clustering, and data visualization.
\textbf{Computing} with large complex data sets, including experience with statistical computing, reproducible analysis, relational database management, and high performance computing techniques such as parallel programming, map-reduce, and cluster computing.
\textbf{Business/scientific communication}, including the ability to quickly acquire domain expertise, contextualize business/scientific problems, translate results into actionable recommendations, and communicate effectively to stakeholders.

\section{Education}
\vspace{-0.08in}
 {\bf University at Buffalo \hfill Aug 2012 -- May 2017 (anticipated)}\\
 Ph.D, Biostatistics \hfill  GPA 3.9/4.0 \\
 Ph.D Thesis advisor: Prof. Marianthi Markatou \\
 M.A. in Biostatistics, conferral date June 15, 2015 \\
 Presidential Fellowship Recipient \\
 {\bf University of the South \hfill Aug 2010 -- May 2012} \\
 Post-baccalaureate student, Mathematics \hfill GPA 4.0/4.0 \\
 {\bf Indiana University, Bloomington    \hfill Aug 2003 -- Aug 2007} \\
 Bachelor of Science in Music and an Outside Field \hfill GPA 3.9/4.0 \\
 Graduated with High Distinction  \\
 Majors: Piano Performance and Psychology (with departmental honors) \\
 Metz Scholar


\section{Experience}

 {\bf University at Buffalo, Dept of Biostatistics}, Buffalo, NY\\
 {\it Research Assistant} \hfill {\bf June 2013 -- Present} \\
 Principal Investigator: Dr. Marianthi Markatou 
 \begin{itemize} \itemsep -2pt
  \item Developed a novel technique for clustering mixed-type data that outperforms competing methods (Foss et al. 2016); implemented in R/Rcpp and Hadoop
  \item Worked with collaborators at the University at Buffalo and IBM Watson Labs to develop algorithms for clustering mixed continuous and categorical data subject to measurement error
  \item Designed, implemented, and analyzed Monte Carlo simulation studies on clusters of up to 200 cores, using R and MPI for parallelization
 \end{itemize}

 {\bf Google}, Mountain View, CA\\
 {\it Tech Intern} \hfill {\bf Summer 2015} \\
 Host: Dr. Ben Davison
 \begin{itemize} \itemsep -2pt
  \item Used theoretical techniques and user data to develop optimal statistical procedures for analyzing user satisfaction
  \item Contributed high-performance algorithms for the analysis and visualization of satisfaction data to internal analysis tools
  \item Gave a talk introducing and justifying these testing procedures to 20+ Google user experience researchers. A 10-year veteran said my talk was one of the best intern presentations he's ever seen.
 \end{itemize}

 {\bf University at Buffalo, Dept of Biostatistics}, Buffalo, NY\\
 {\it Teaching Assistant} \hfill {\bf Fall 2012, Spring 2013}
 \begin{itemize} \itemsep -2pt
  \item TA for graduate and undergraduate statistics courses
  \item Received excellent student evaluations: 94\% agreement with the statement ``Presents material well,'' 80\% respondents categorizing overall teaching effectiveness as ``One of the best'' or ``Above Average'', and comments such as ``One of the best TAs I have had so far.''
 \end{itemize}
 

 {\bf Yale University School of Medicine}, New Haven, CT\\
 {\bf Children's Hospital of Philadelphia}, Philadelphia, PA\\
 {\bf Developmental Neuroimaging Lab}\\
 Principal Investigator: Dr. Robert Schultz\\
 {\it Research Assistant (full time)} \hfill {\bf Jul 2007 -- Jun 2009}
 \begin{itemize} \itemsep -2pt
  \item Assisted in the design and analysis of fMRI studies of
        visual perception and social cognition
 \end{itemize}

 

\section{Selected Publications}
%\hspace{\listInd} \fullcite{foss16a}
\textbf{AH Foss}, M Markatou, B Ray, and A Heching (2016).
 A Semiparametric Method for Clustering Mixed Data,
 \textit{Machine Learning}
 105(\textbf{3}), 419--458. \\
 \vspace{-0.15in}

%\hspace{\listInd} \fullcite{foss13}
 \textbf{AH Foss}, PK Duffner, and RL Carter (2013).
 Lifetime Risk Estimators in Epidemiological Studies of Krabbe Disease: Review and Monte Carlo Comparison,
 \textit{Rare Diseases}
 1(\textbf{2}), e25212. \\
 \vspace{-0.15in}

%\hspace{\listInd} \fullcite{foss07}
 \textbf{AH Foss}, EL Altschuler, and KH James (2007).
 Neural Correlates of Pythagorean Ratio Rules,
 \textit{Neuroreport}
 18, 1521--1525. \\
 \vspace{-0.15in}

%\hspace{\listInd} \fullcite{foss16b}
 \textbf{AH Foss} and M Markatou (under review).
  Clustering Mixed-Type Data in R and Hadoop,
software available on  \url{https://CRAN.R-project.org/package=kamila} and \url{https://github.com/ahfoss/kamila}.


\vspace{-0.08in}

\section{Presentations}
%\hspace{\listInd} \fullcite{jsm15}
\textbf{AH Foss}, M Markatou, B Ray, and A Heching (2015).
 A Semiparametric Method for Clustering Mixed Data,
 \textit{Joint Statistical Meetings}, Seattle, WA, USA. \\
 \vspace{-0.15in}

%\hspace{\listInd} \fullcite{google15}
 \textbf{AH Foss} (2015).
 Clustering Mixed Continuous and Categorical Data,
 \textit{Google Statistics Journal Club}, Mountain View, CA, USA. \\
 \vspace{-0.15in}

%\hspace{\listInd} \fullcite{isbis14} (Invited Presentation)
\textbf{AH Foss}, M Markatou, B Ray, and A Heching (2014).
 Clustering Mixed Data Subject to Measurement Error,
 \textit{International Society for Business and Industrial Statistics}, ASA Section on Statistical Learning and Data Mining, Durham, NC, USA.


\section{Academic  Honors} 
\begin{itemize} \itemsep -2pt
 \item Two Honorable Mentions, NSF GRF Program \hfill Spring 2012 and 2013
 \item Perry Poster Award, University at Buffalo \hfill April 19, 2013
 \item Presidential Fellowship, University at Buffalo (\$23,000) \hfill Fall 2012
 \item Excellence in Research Award, IU Psychology Dept \hfill April 19, 2007
 \item Capstone Grant, Howard Hughes Medical Institute (\$4,250) \hfill Spring 2006
 \item Metz Scholarship, IU Honors College (\$56,000) \hfill Fall 2003
 \item Merit Scholarship, IU School of Music (\$40,000) \hfill Fall 2003
\end{itemize}


\section{Programming Skills}
\textbf{Fluent} (\textit{Used daily; I can write substantial programs without consulting references}): \\
  R, \LaTeX{} markup language, dynamic report generation with knitr (PDF, HTML, or MS word output) 

\textbf{Formerly Fluent} (\textit{I achieved fluency at some point within the last 10 years}):  Java, MATLAB, Python

\textbf{Familiar} (\textit{I can use at a high level while consulting references}):  C, Go, Hadoop, HTML/CSS, Mathematica, Processing, SAS, SQL, Visual Basic

%\section{Other Computing Skills}
%  \textbf{Software}: SPSS, VIM, MS Office Suite
%
%  \textbf{Operating systems}: Linux (Red Hat, Ubuntu, CentOS) and Windows (98/XP/Vista/7/8)
%
%  \textbf{Shell programming} (BASH)
%
%  \textbf{Batch scheduling systems}: SLURM, SGE
%


\section{Selected Coursework}
  \textbf{University at Buffalo, Master's Level}: Regression Analysis, Categorical Data Analysis, Multivariate Data Analysis, Statistics for Bioinformatics \\
 \textbf{University at Buffalo, PhD Level}: Advanced Modeling, Advanced Categorical Data Analysis, Advanced Survival Analysis, Theory of Linear Models, Limit Theory, Design and Analysis of Observational Studies \\


\end{document} 


