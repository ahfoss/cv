
\hyphenation{pdfTeX Mathematica}

\documentclass{article}

% capitalize section titles; center
\makeatletter
\renewcommand\section{\@startsection {section}{1}{\z@}%
                                   {-2.5ex \@plus -1ex \@minus -.2ex}%beforeskip
                                   {1.0ex \@plus.2ex}%afterskip
                                   {\Large\scshape\centerline}*}
\makeatother

\setlength{\parindent}{0cm}

\newcommand\listInd{0.4cm}

\usepackage[inner=1.0in,outer=1.0in,bottom=2.5cm,top=2.5cm]{geometry} % control borders

\usepackage{xcolor} % for colors
\definecolor{darkgray}{gray}{0.4}
\definecolor{lightgray}{gray}{0.7}

\usepackage{fancyhdr} % for headers
\usepackage{lastpage}
\pagestyle{fancy}

% clear default headers/footers
\fancyhead{}
\fancyfoot{}

\fancyfoot[L]{\textcolor{darkgray}{Alexander Foss}}
\fancyfoot[C]{\textcolor{darkgray}{\thepage\ of \pageref{LastPage}}}
\fancyfoot[R]{\textcolor{darkgray}{alexanderhfoss@gmail.com}}

% remove line
\renewcommand{\headrulewidth}{0pt}
\renewcommand{\footrulewidth}{0.4pt}
\renewcommand{\footrule}{\hbox to\headwidth{ \color{lightgray}\leaders\hrule height \footrulewidth\hfill}}

\usepackage[backend=biber,style=authoryear,maxbibnames=99,maxcitenames=99]{biblatex}
\addbibresource{../biblio/resumeBiblio.bib}


%\textwidth=5.2in % increase textwidth to get smaller right margin
%\usepackage{helvetica} % uses helvetica postscript font (download helvetica.sty)
%\usepackage{newcent}   % uses new century schoolbook postscript font 

\begin{document} 

\thispagestyle{empty}

% suppress all page numbers
%\pagenumbering{gobble}
%\pagestyle{empty} % suppresses all page numbers, headers, and footers
 
 \begin{centering}
\Large
Alexander H. Foss
\vspace{0.05in}
\normalsize

\hspace*{\fill}
alexanderhfoss@gmail.com
\hfill \textbullet
\hfill (931) 636-8109 
\hfill \textbullet
\hfill linkedin.com/in/alex-foss-30939383
\hspace*{\fill}

%\hfill github.com/ahfoss
%\hspace*{\fill}

\end{centering}

\noindent\makebox[\linewidth]{\rule{\textwidth}{0.4pt}}

 
%\section{Summary}
%I am an inquisitive, effective, and creative statistician and data scientist with extensive skills in multiple areas:
%\textbf{Technical Leadership} 4+ years experience leading technically diverse teams of 5+ Masters and PhD level staff in addressing complex and open-ended customer problems.
%\textbf{Statistics and machine learning}, including regression, hypothesis testing, classification, clustering, and natural language processing using a range of classical and deep learning techniques.
%\textbf{Technical communication}, including presentations at international statistics conferences and to colleagues at institutions such as Google, Yale, and the Children's Hospital of Philadelphia.
%\textbf{Computing}, including experience with various programming languages (including R, python, BASH, SQL, C, and Go) and experience with high performance computing techniques such as parallel programming, map-reduce, and cluster computing.
%
 
\section{Education}
 {\bf University at Buffalo \hfill Aug 2012 -- May 2017 }\\
 Ph.D, Biostatistics \hfill  GPA 3.9/4.0\\
 Ph.D Thesis advisor: Prof. Marianthi Markatou \\
 M.A. in Biostatistics, conferral date June 15, 2015 \\
 {\bf University of the South \hfill Aug 2010 -- May 2012}\\
 Post-baccalaureate student, Mathematics \hfill GPA 4.0/4.0\\
 {\bf University of California, Berkeley \hfill Aug 2009 -- Mar 2010}\\
 Ph.D Candidate, Psychology \hfill GPA 4.0/4.0\\
 {\bf Indiana University, Bloomington    \hfill Aug 2003 -- Aug 2007}\\
 Bachelor of Science in Music and an Outside Field \hfill GPA 3.9/4.0\\
 Graduated with High Distinction  \\
 Major: Piano Performance\\ 
 Outside Field: Psychological and Brain Sciences, with Departmental Honors


\section{Experience}

 {\bf Sandia National Laboratories}, Albuquerque, NM\\
 {\it Principal Member of the Technical Staff} \hfill {\bf June 2017 -- present} \\
 Organization: Statistical Sciences
 \begin{itemize} \itemsep -2pt
  \item Led technically diverse teams of staff members and graduate interns on statistics and machine learning research topics related to national security
  \item Designed novel algorithms for classification, cluster analysis, streaming data analysis, and semisupervised data analysis
  \item Numerous meetings and briefings to government stakeholders to identify business opportunities and present solutions
 \end{itemize}

 {\bf University at Buffalo, Dept of Biostatistics}, Buffalo, NY\\
 {\it Research Assistant} \hfill {\bf June 2013 -- May 2017} \\
 Principal Investigator: Dr. Marianthi Markatou 
 \begin{itemize} \itemsep -2pt
  \item Developed a novel technique for clustering mixed-type data that outperforms competing methods (Foss et al. 2016) implemented in R/C++ and Hadoop
  \item Worked with collaborators at the University at Buffalo and IBM Watson Labs to develop algorithms for clustering mixed continuous and categorical data subject to measurement error
  \item Designed, implemented, and analyzed Monte Carlo simulation studies on large compute clusters, using R, SQL for data management, and MPI for parallelization
 \end{itemize}

 {\bf Google}, Mountain View, CA\\
 {\it Tech Intern} \hfill {\bf Summer 2015} \\
 Host: Dr. Ben Davison
 \begin{itemize} \itemsep -2pt
  \item Used theoretical techniques and user data to develop optimal statistical procedures for analyzing user satisfaction
  \item Contributed high-performance algorithms for the analysis and visualization of satisfaction data to internal analysis tools
  \item Gave a talk introducing and justifying these testing procedures to 20+ Google user experience researchers. A 10-year veteran said my talk was one of the best intern presentations he's ever seen.
 \end{itemize}

 {\bf University at Buffalo, Dept of Biostatistics}, Buffalo, NY\\
 {\it Teaching Assistant} \hfill {\bf Fall 2012, Spring 2013}
 \begin{itemize} \itemsep -2pt
  \item TA for graduate and undergraduate statistics courses
  \item Led weekly recitation sections, held weekly office hours, graded homework, assisted in grading exams
  \item Received excellent student evaluations: 94\% agreement with the statement ``Presents material well,'' 80\% respondents categorizing overall teaching effectiveness as ``One of the best'' or ``Above Average'', and comments such as ``One of the best TAs I have had so far.'' (Spring 2013)
 \end{itemize}
 
 {\bf University at Buffalo, Dept of Biostatistics}, Buffalo, NY\\
 {\bf Population Health Observatory}\\
 {\it Research Assistant} \hfill {\bf Summers 2010, 2011, 2012} \\
 Principal Investigator: Dr. Randolph Carter
 \begin{itemize} \itemsep -2pt
  \item Implemented a Monte Carlo simulation in R evaluating published 
        methods of calculating the lifetime risk at birth of Krabbe disease
        (published 2013)
  \item Assisted in the writing of grant proposals to the NIDDK and HRSA,
        as well as an ARRA grant proposal.
 \end{itemize}

 {\bf New York University, Dept of Applied Psychology}, New York, NY\\
 Principal Investigator: Dr. Arnold Grossman\\
 {\it Data Analyst (part-time)} \hfill {\bf Mar 2011 -- June 2012}
 \begin{itemize} \itemsep -2pt
  \item Analyzed associations between domestic abuse, neglect and
        other variables in elderly LGBT adults, as well as associations
        between homelessness and traumatic life events in LGBT youth
  \item Co-author on a paper investigating the relationship between
        pubertal timing and sexual identity development (published 2014)
 \end{itemize}

 {\bf University of California, Dept of Psychology}, Berkeley, CA\\
 {\bf Affective Cognitive Neuroscience Lab}\\
 Principal Investigator: Dr. Sonia Bishop\\
 {\it Graduate Student Researcher} \hfill {\bf Aug 2009 -- Apr 2010}
 \begin{itemize} \itemsep -2pt
  \item Assisted in the design and analysis of a functional MRI study 
        investigating neural processing of ambiguous and pure emotions 
  \item Assisted in the writing of an NIMH BRAINS research grant concerning
        anxiety reduction biofeedback training using real-time fMRI
 \end{itemize}

 {\bf Yale University School of Medicine}, New Haven, CT\\
 {\bf Children's Hospital of Philadelphia}, Philadelphia, PA\\
 {\bf Developmental Neuroimaging Lab}\\
 Principal Investigator: Dr. Robert Schultz\\
 {\it Research Assistant (full time)} \hfill {\bf Jul 2007 -- Jun 2009}
 \begin{itemize} \itemsep -2pt
  \item Assisted in the design and analysis of fMRI studies of
        visual perception and social cognition
 \end{itemize}

 {\bf Indiana University, Dept of Psychological \& Brain Sciences}, Bloomington, IN\\
 {\bf Cognition \& Action Neuroimaging Lab}\\
 Principal Investigator: Dr. Karin James\\
 {\it Undergraduate Research Assistant} \hfill {\bf Sep 2005 -- Aug 2007}
 \begin{itemize} \itemsep -2pt
  \item Designed, analyzed, and published results of an fMRI experiment 
        investigating the neural correlates of auditory perception of tone 
	combinations; analysis conducted using BrainVoyager QX (published 2007)
 \end{itemize}
 

\section{Publications}
%\textbf{Submitted:}

\textbf{Published/Accepted:} 

\hspace{\listInd} \fullcite{tu21}

\hspace{\listInd} \fullcite{fossHitting}

\hspace{\listInd} \fullcite{foss19}

\hspace{\listInd} \fullcite{foss18}. Software available at \url{https://github.com/ahfoss/kamila} and on CRAN.

\hspace{\listInd} \fullcite{foss16a}

\hspace{\listInd} \fullcite{grossman14}

\hspace{\listInd} \fullcite{foss13}

\hspace{\listInd} \fullcite{bar12}

\hspace{\listInd} \fullcite{foss07} \\

\textbf{In progress:} 

%\hspace{\listInd} \fullcite{fossDisc}

\hspace{\listInd} \fullcite{fossMedea}

%\hspace{\listInd} \fullcite{fossConvergence}

\section{Presentations} 

\hspace{\listInd} \fullcite{jsm23}

\hspace{\listInd} \fullcite{upstat22}

\hspace{\listInd} \fullcite{jsm21}

\hspace{\listInd} \fullcite{jsm19}

\hspace{\listInd} \fullcite{abqasa19}

\hspace{\listInd} \fullcite{enar17}

\hspace{\listInd} \fullcite{jsm15}

\hspace{\listInd} \fullcite{google15}

\hspace{\listInd} \fullcite{isbis14} (Invited Presentation)


\section{Posters}

\hspace{\listInd} \fullcite{coda23}

\hspace{\listInd} \fullcite{jsm13}

\hspace{\listInd} \fullcite{vss10}

\hspace{\listInd} \fullcite{imfar10}

\hspace{\listInd} \fullcite{hbm06}




\section{Academic Honors} 
\begin{itemize} \itemsep -2pt
 \item Two Honorable Mentions, NSF GRF Program \hfill Spring 2012 and 2013
 \item Perry Poster Award, University at Buffalo \hfill April 19, 2013
 \item Presidential Fellowship, University at Buffalo (\$23,000) \hfill Fall 2012
 \item Diebold Fellowship, UC Berkeley Psychology Dept (\$14,600) \hfill Fall 2009
 \item Excellence in Research Award, IU Psychology Dept \hfill April 19, 2007
 \item Honors Thesis Award, IU Honors College \hfill Spring 2007
 \item Capstone Grant, Howard Hughes Medical Institute (\$4,250) \hfill Spring 2006
 \item Metz Academic Merit Scholarship, IU Honors College (\$56,000) \hfill Fall 2003
 \item Merit Scholarship, IU School of Music (\$40,000) \hfill Fall 2003
\end{itemize}
 

%% Tabulate computer skills; p{3in} defines paragraph 3 inches wide
%\section{Computer Skills}
%
%\textbf{Languages:} R, Bash, C, Java, Go, Mathematica, MATLAB, 
%                             Processing, Python, Visual Basic, HTML,
%                             CSS
%
%\textbf{Software:} SPSS, SAS, pdf\TeX{}, 
%                         Sweave, VIM, 
%                         OpenOffice/LibreOffice, MS Office Suite, Hadoop
%
%\textbf{Operating Systems:} Windows (98/XP/Vista/7/8),
%                                     Macintosh (OS9, OSX),
%                                     Linux (Red Hat, Ubuntu, CentOS)


\section{Selected Coursework}
  \textbf{University at Buffalo, PhD Level}: Topics in Advanced Modeling, Advanced Categorical Data Analysis, Advanced Survival Analysis, Theory of Linear Models, Limit Theory, Theory of Statistical Inference, Design and Analysis of Observational Studies \\
  \textbf{University at Buffalo, Master's Level}: Regression Analysis, Categorical Data Analysis, Multivariate Data Analysis, Statistics for Bioinformatics, Statistical Comparisons and Associations \\
  \textbf{University of the South}: Multidimensional Calculus, Linear Algebra, Discrete Mathematical Structures, Genomics, Numerical Analysis \\
  \textbf{University of California, Berkeley}: fMRI Methods, Proseminar in Cognition, Brain, and Behavior \\
  \textbf{Indiana University, Bloomington}: Molecular Biology, Abnormal Psychology, Behavioral Neuroscience, Lab in Behavioral Neuroscience, Lab in Neuroimaging Methods 

%\section{Service/Extracurricular}
%{\bf University at Buffalo Computational Sciences Club} \\
%{\it Member, Volunteer Speaker} \hfill {\bf Summer 2016 -- 2017}
%\begin{itemize} \itemsep -2pt
%  \item Gave a talk on June 1, 2016 entitled ``Clustering Mixed Continuous and Categorical Data''
%\end{itemize}
%
%{\bf University at Buffalo School of Public Health and Health Professions (SPHHP), Academic Affairs Committee} \\
%{\it Committee Member} \hfill {\bf Sep 2014 -- Present}
%\begin{itemize} \itemsep -2pt
%  \item Worked with the Senior Associate Dean for Academic and Student Affairs to establish standards for undergraduate, graduate, and post graduate study in the SPHHP
%  \item Issues addressed included the establishment of new programs, new course proposals, academic policies and procedures, admissions, and clinical education policies
%\end{itemize}
%
%{\bf New York State Center of Excellence in Bioinformatics and Life Sciences} \\
% {\it Volunteer Speaker} \hfill {\bf July 24, 2012}
% \begin{itemize} \itemsep -2pt
%  \item Gave a presentation on face recognition, brain imaging, and autism to a group of exchange students
% \end{itemize}
%
%{\bf The Franklin Institute Museum of Science}, Philadelphia, PA\\
%{\it Volunteer} \hfill {\bf Nov 2007 -- Oct 2008}
% \begin{itemize} \itemsep -2pt
%  \item Conducted science demonstrations including paper-making, the ``brain bar'' neuroscience exhibit, and the Baldwin 60,000 steam engine simulation
%  \item Interacted individually with children ages 4--16 and their parents
% \end{itemize}
%
%{\bf Wonderlab Museum of Science}, Bloomington, IN\\
%{\it Volunteer} \hfill {\bf May 2006 -- Jul 2007}
% \begin{itemize} \itemsep -2pt
%  \item Conducted ``science-on-the-spot'' demonstrations
%  \item Interacted individually with children ages 4--12 and their parents
% \end{itemize}
%
%{\bf International Service Learning}, Kansas City, MO/San Jose, Costa Rica\\
%{\it Volunteer} \hfill {\bf Mar 10 -- 18, 2007}
% \begin{itemize} \itemsep -2pt
%  \item Helped set up clinics in Costa Rica in the towns of Tib\'{a}s and Puntarenas
%  \item Conducted community triage, took patient histories, gave eye examinations, and prescribed reading glasses
% \end{itemize}


\end{document} 


